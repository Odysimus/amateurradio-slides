\documentclass[12pt,letterpaper]{report}
\usepackage[utf8]{inputenc}
\usepackage{amsmath}
\usepackage{amsfonts}
\usepackage{amssymb}
\usepackage{graphicx}
\usepackage[width=20.00in, height=15.00in, left=0.25in, right=0.25in, top=0.25in, bottom=0.25in]{geometry}
\author{Anthony Odenthal, KE7OSN Amateur Extra}
\title{Presentation Notes for the Technician Class Presentation}
\begin{document}
\section*{Preface}
These notes are intended to assist in the giving of the presentation. They should help the presenter cover topics since much of the content of the presentation is designed to touch on topics and have the presenter lecture in more detail. It would also not hurt for the presenter to have read, or have available a copy of the ARRL license manual.

This document is broken into sections that follow the same structure as the presentation.

\section*{Introduction}
Here you should introduce yourself and what amateur radio is. This would be a good time to let your students know how and why you got into amateur radio. Explain how the exam is run, and remind students to not stress.

\section*{FCC rules, descriptions, and definitions}
This covers mostly the administrative issues.

\subsection*{Who's In Charge}
In this part we cover how amateur radio has a good deal of self-regulation and that is done within the framework of the FCC, and that the FCC in turn works within the rules of international treaty, particularly the ITU. Explain what the ITU, FCC, and Frequency Coordinators are. A good example is the 2 meter band, the ARRL band plan carves up the band into small chunks for specific purposes, while the FCC doesn't go into that level of detail.

\subsection*{Part 97}
Next up is more detail on Part 97, here you can explain that for the most part this is where the rules can be found. This presentation emphasizes the noncommercial aspect. Some of the rules are there to protect the amateur way of life, and others are comprises on issues namely stemming from the world wars and the cold war.

\subsection*{Licenses}
License classes, you may go into details about what the novice and advanced were, and how complicated the licensing system was. Also mentioned are types of license, point out in other services the stations and operator licenses are separated. This is also a good time to let advise students that they are not allowed to transmit until they show up in the ULS, but if they upgrade later they can use those upgraded privileges as soon as they pass.

\subsection*{Callsigns}
Go over what makes a valid amateur callsign, a valid us callsign, and a couple other types of calls. Advise that the control operator is the \textit{licensed} operator in charge of the station, it is the control operator's callsign and license that are being used. This presentation very briefly touches on special cases, depending on the students it may or may not be useful to explain these cases in more details.
For an individual working a station whose station license is a lower license than the operator (e.g.\ an extra at a club whose trustee is a general) the ID should be ``station call''/``operator's call''
For a 1x1 the normal callsign should be transmitted at the start or operations (change of frequency), the end of operations (change of frequency), change of normal callsign (change of operators, unless a club call is used), and once every hour. The Id should take the form of ``normal call''/``1x1''
For an operator who has upgraded but the upgrade hasn't shown in the database they should append an ``AG'' for general or ``AE'' for extra to their call, when using the privileges granted to the new class and not the old one. E.g.\ An upgrade to extra from general on 14.175MHz would id as ``callsign''/AE but on 14.300MHz would just ID with their call.

\subsection*{Frequencies}
This is where you should talk about how we share a few of the bands with other users, in mosts cases we are primary, but in a few cases, 30m, 60m,1.25m we have restrictions.

\section*{Station operation; choosing an operating frequency, calling another station, test transmissions, use of minimum power, frequency use, band plans}

\end{document}

\documentclass[12pt,letterpaper,twocolumn]{report}
\usepackage[utf8]{inputenc}
\usepackage{amsmath}
\usepackage{amsfonts}
\usepackage{amssymb}
\usepackage{graphicx}
\usepackage[width=20.00in, height=15.00in, left=0.50in, right=0.50in, top=0.50in, bottom=0.50in]{geometry}
\author{Anthony Odenthal}
\begin{document}

\section{Section 1 6 questions}
\textbf{T1A01 [97.3(a)(4)]}\\
For whom is the Amateur Radio Service intended?\\
A. Persons who have messages to broadcast to the public\\
B. Persons who need communications for the activities of their immediate family members, relatives and friends\\
C. Persons who need two-way communications for personal reasons\\
D. Persons who are interested in radio technique solely with a personal aim and without pecuniary interest\\


\textbf{T1B01 [97.3(a)(28)]}\\
What is the ITU?\\
A. An agency of the United States Department of Telecommunications Management\\
B. A United Nations agency for information and communication technology issues\\
C. An independent frequency coordination agency\\
D. A department of the FCC\\

\textbf{T1C02}\\
Which of the following is a valid US amateur radio station call sign?\\
A. KMA3505\\
B. W3ABC\\
C. KDKA\\
D. 11Q1176\\

\textbf{T1D02 [97.111(a)(5)]}\\
On which of the following occasions may an FCC-licensed amateur station exchange messages with a U.S.\\ military station?\\
A. During an Armed Forces Day Communications Test\\
B. During a Memorial Day Celebration\\
C. During an Independence Day celebration\\
D. During a propagation test\\

\textbf{T1E02 [97.7(a)]}\\
Who is eligible to be the control operator of an amateur station?\\
A. Only a person holding an amateur service license from any country that belongs to the United Nations\\
B. Only a citizen of the United States\\
C. Only a person over the age of 18\\
D. Only a person for whom an amateur operator/primary station license grant appears in the FCC database or who is authorized for alien reciprocal operation\\

\textbf{T1F02 [97.119 (a)]}\\
When using tactical identifiers, how often must your station transmit the station's FCC-assigned call sign? \\
A. Never, the tactical call is sufficient\\
B. Once during every hour\\
C. Every ten minutes\\
D. At the end of every communication\\

\section{Section 2 3 questions}

\textbf{T2A03}\\
What is a common repeater frequency offset in the 70 cm band?\\
A. Plus or minus 5 MHz\\
B. Plus or minus 600 kHz\\
C. Minus 600 kHz\\
D. Plus 600 kHz\\

\textbf{T2B01}\\
What is the term used to describe an amateur station that is transmitting and receiving on the same frequency?\\
A. Full duplex communication\\
B. Diplex communication\\
C. Simplex communication\\
D. Half duplex communication\\

\textbf{T2C07}\\
What should you do to minimize disruptions to an emergency traffic net once you have checked in?\\
A. Whenever the net frequency is quiet, announce your call sign and location\\
B. Move 5 kHz away from the net's frequency and use high power to ask other hams to keep clear of the net frequency\\
C. Do not transmit on the net frequency until asked to do so by the net control station\\
D. Wait until the net frequency is quiet, then ask for any emergency traffic for your area \\

\section{Section 3 3 questions}

\textbf{T3A02}\\
Why are UHF signals often more effective from inside buildings than VHF signals?\\
A. VHF signals lose power faster over distance\\
B. The shorter wavelength allows them to more easily penetrate the structure of buildings\\
C. This is incorrect; VHF works better than UHF inside buildings\\
D. UHF antennas are more efficient than VHF antennas\\

\textbf{T3B03}\\
What are the two components of a radio wave?\\
A. AC and DC\\
B. Voltage and current\\
C. Electric and magnetic fields\\
D. Ionizing and non-ionizing radiation\\

\textbf{T3C02}\\
Which of the following might be happening when VHF signals are being received from long distances?\\
A. Signals are being reflected from outer space\\
B. Signals are arriving by sub-surface ducting\\
C. Signals are being reflected by lightning storms in your area\\
D. Signals are being refracted from a sporadic E layer\\

\section{Section 4 2 questions}

\textbf{T4A08}\\
Which type of conductor is best to use for RF grounding?\\
A. Round stranded wire \\
B. Round copper-clad steel wire\\
C. Twisted-pair cable\\
D. Flat strap\\

\textbf{T4B02}\\
Which of the following can be used to enter the operating frequency on a modern transceiver?\\
A. The keypad or VFO knob\\
B. The CTCSS or DTMF encoder\\
C. The Automatic Frequency Control\\
D. All of these choices are correct\\

\section{Section 5 4 questions}

\textbf{T5A01}\\
Electrical current is measured in which of the following units?\\
A. Volts\\
B. Watts\\
C. Ohms\\
D. Amperes\\

\textbf{T5B05}\\
Which of the following is equivalent to 500 milliwatts?\\
A. 0.02 watts\\
B. 0.5 watts\\
C. 5 watts\\
D. 50 watts\\

\textbf{T5C02}\\
What is the ba7sic unit of capacitance?\\
A. The farad\\
B. The ohm\\
C. The volt\\
D. The henry\\

\textbf{T5D04}\\
What is the resistance of a circuit in which a current of 3 amperes flows through a resistor connected to 90 volts?\\
A. 3 ohms\\
B. 30 ohms\\
C. 93 ohms\\
D. 270 ohms\\

\section{Section 6 4 questions}

\textbf{T6A02}\\
What type of component is often used as an adjustable volume control?\\
A. Fixed resistor\\
B. Power resistor\\
C. Potentiometer\\
D. Transformer\\

\textbf{T6B07}\\
What does the abbreviation "LED" stand for?\\
A. Low Emission Diode\\
B. Light Emitting Diode\\
C. Liquid Emission Detector\\
D. Long Echo Delay\\

\textbf{T6C01}\\
What is the name for standardized representations of components in an electrical wiring diagram?\\
A. Electrical depictions\\
B. Grey sketch\\
C. Schematic symbols\\
D. Component callouts\\

\textbf{T6D06}\\
What component is commonly used to change 120V AC house current to a lower AC voltage for other uses?\\
A. Variable capacitor\\
B. Transformer\\
C. Transistor\\
D. Diode\\

\section{Section 7 4 questions}

\textbf{T7A03}\\
What is the function of a mixer in a superheterodyne receiver?\\
A. To reject signals outside of the desired passband\\
B. To combine signals from several stations together\\
C. To shift the incoming signal to an intermediate frequency\\
D. To connect the receiver with an auxiliary device, such as a TNC\\

\textbf{T7B05}\\
What is a logical first step when attempting to cure a radio frequency interference problem in a nearby telephone?\\
A. Install a low-pass filter at the transmitter\\
B. Install a high-pass filter at the transmitter\\
C. Install an RF filter at the telephone\\
D. Improve station grounding\\

\textbf{T7C04}\\
What reading on an SWR meter indicates a perfect impedance match between the antenna and the feedline?\\
A. 2 to 1\\
B. 1 to 3\\
C. 1 to 1\\
D. 10 to 1\\

\textbf{T7D08}\\
Which of the following types of solder is best for radio and electronic use?\\
A. Acid-core solder\\
B. Silver solder\\
C. Rosin-core solder\\
D. Aluminum solder\\

\section{Section 8 4 questions}

\textbf{T8A03}\\
Which type of voice modulation is most often used for long-distance or weak signal contacts on the VHF and UHF bands?\\
A. FM\\
B. AM\\
C. SSB\\
D. PM\\

\textbf{T8B04}\\
Which amateur stations may make contact with an amateur station on the International Space Station using 2 meter and 70 cm band amateur radio frequencies?\\
A. Only members of amateur radio clubs at NASA facilities\\
B. Any amateur holding a Technician or higher class license\\
C. Only the astronaut's family members who are hams\\
D. You cannot talk to the ISS on amateur radio frequencies\\

\textbf{T8C03}\\
What popular operating activity involves contacting as many stations as possible during a specified period of time?\\
A. Contesting\\
B. Net operations\\
C. Public service events\\
D. Simulated emergency exercises\\

\textbf{T8D09}\\
What code is used when sending CW in the amateur bands?\\
A. Baudot\\
B. Hamming\\
C. International Morse\\
D. Gray\\

\section{Section 9 2 questions}

\textbf{T9A04}\\
What is a disadvantage of the "rubber duck" antenna supplied with most handheld radio transceivers?\\
A. It does not transmit or receive as effectively as a full-sized antenna\\
B. It transmits a circularly polarized signal\\
C. If the rubber end cap is lost it will unravel very quickly\\
D. All of these choices are correct\\

\textbf{T9B06}
Which of the following connectors is most suitable for frequencies above 400 MHz?\\
A. A UHF (PL-259/SO-239) connector\\
B. A Type N connector\\
C. An RS-213 connector\\
D. A DB-23 connector\\


\section{Section 0 3 questions}

\textbf{T0A02}\\
How does current flowing through the body cause a health hazard?\\
A. By heating tissue\\
B. It disrupts the electrical functions of cells\\
C. It causes involuntary muscle contractions\\
D. All of these choices are correct\\

\textbf{T0B04}\\
Which of the following is an important safety precaution to observe when putting up an antenna tower?\\
A. Wear a ground strap connected to your wrist at all times\\
B. Insulate the base of the tower to avoid lightning strikes\\
C. Look for and stay clear of any overhead electrical wires\\
D. All of these choices are correct\\

\textbf{T0C02}\\
Which of the following frequencies has the lowest Maximum Permissible Exposure limit?\\
A. 3.5 MHz\\
B. 50 MHz\\
C. 440 MHz\\
D. 1296 MHz\\

\newpage
\onecolumn
\section{Answers}
D, B, B, A, D, C; B, C, D; B, C, D; D, A; D, B, A, B; C, B, C, B; C, C, C, C; C, B, A, C; A, B; D, C, B.
\end{document}
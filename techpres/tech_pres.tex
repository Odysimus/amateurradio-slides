\documentclass[10pt]{beamer}
\usepackage[utf8]{inputenc}
\usepackage{amsmath}
\usepackage{amsfonts}
\usepackage{amssymb}
\usepackage{graphicx}
\usepackage{alltt}

\graphicspath{{../sharedfiles/}}

%\usepackage[size=custom,width=27,height=15,scale=1]{beamerposter}
\author{Anthony Odenthal, KE7OSN Amateur Extra}
\title{Technician and General Class Amateur Radio \& Satellite Stuff}
\AtBeginSection[]
{
\begin{frame}
\frametitle{Table of Contents}
\tableofcontents[currentsection]
\end{frame}
}
\AtBeginSubsection[]
{
  \begin{frame}
    \frametitle{Table of Contents}
    \tableofcontents[currentsection,currentsubsection]
 \end{frame}
}

\usetheme{PaloAlto}
\usecolortheme{beetle}
\usepackage{mdwlist}

\begin{document}

\frame{\titlepage}

%\begin{frame}
%\frametitle{Table of Contents}
%\tableofcontents
%\end{frame}

\section{Introduction}

\begin{frame}
\frametitle{Welcome}
Welcome, over the next several sessions we will cover a substantial amount of information. please ask questions and slow me down.\\
The goals are:
\begin{itemize}
\item To introduce you to Amateur Radio \pause
\item Prepare you to take (and pass) the technician and general exams \pause
\item Introduce you to satellite communications.
\end{itemize}
\end{frame}

\begin{frame}
\frametitle{A little about myeself}
\begin{itemize}
%\item I'm a student at Oregon State University %Going to keep to just radio stuff
\item Passed Tech Sept 2007
\item Passed Gen Oct 2007
\item Joined Benton County ARES April 2012
\item Passed Extra April 2012
\item Became a VE in June 2012
\end{itemize}
\end{frame}

\begin{frame}
\frametitle{What is Amateur Radio?}
Amateur radio are people and activities that are regulated and encouraged, in the US and abroad, that allow licensed individuals to play around with radio waves, electronics, software, techniques, practices, and equipment to do all sorts of really cool stuff. Radio Amateurs are some of the least restricted users of radio spectrum, and with that freedom they have proven time and time again their worth.
The term Amateur refers to someone who does something as a pastime rather than a profession.
\end{frame}

\begin{frame}
\frametitle{Some useful tools}
Some things you may want to look into as useful for studying
\begin{itemize}
\item AA9PW practice exams \url{http://aa9pw.com}
\item ARRL license Manuals \url{http://www.arrl.org/shop/Licensing-Education-and-Training/}
\end{itemize}
\end{frame}

\begin{frame}
\frametitle{About the test}
\begin{itemize}
\item 35 questions \pause
\item Multiple Choice \pause
\item No time limit \pause
\item 396 questions in the tech pool, 457 in the general \pause
\item Need a 75\% to pass
\end{itemize}
\end{frame}

\begin{frame}
\frametitle{Shal we begin?}
Remember if I go too fast or you have questions, let me know.
\end{frame}

\section{General Rules}
%\subsection{T1A - Amateur Radio services; purpose of the amateur service, amateur-satellite service, operator/primary station license grant, where FCC rules are codified, basis and purpose of FCC rules, meanings of basic terms used in FCC rules}
\subsection{Who's In Charge}
\begin{frame}
\frametitle{Who's In charge}
International Telecommunications Union (ITU)
\begin{itemize}
\item Worldwide, treaty-based organization that allocates frequencies for specific uses.
\item Primary Users - first "rights" to a frequency
\item Secondary Users - permitted to use a frequency but must not interfere with a primary user
\item World divided into 3 regions, US is in Region 2
\item Creates "bands" - sections of spectrum allocated for amateur radio use.
\end{itemize}
\end{frame}

\begin{frame}
\frametitle{Who's In Charge}
Federal Comunications Commission (FCC)
\begin{itemize}
\item Promulgates  rules for non-federal radio users within ITU spec
\item Divides amateur bands into mode-specific sub-bands
\item Rules for telecommunications are in the Code of Federal Regulations, Chapter 47
\item Rules for amateur radio are in Part 97 of Chapter 47 (47 CFR 97)
\end{itemize}
\end{frame}

\begin{frame}
\frametitle{Who's In Charge}
Frequency Coordinator
\begin{itemize}
\item FCC recognized regional groups that coordinate the use of bands between large number of users \pause
\item Appointed by amateurs for amateurs \pause
\item Intended to help reduce and allow resolution of interference issues \pause
\item Voluntary rules unless there is interference, then the coordinated user "wins"\pause
\item Gentleman's agreement
\end{itemize}
\end{frame}

\begin{frame}
\frametitle{FCC allocations}
\begin{center}
\includegraphics[width=\textwidth]{2003-allochrt.pdf}
\end{center}
\end{frame}

\subsection{Part 97}

\begin{frame}
\frametitle{47 CRF 97.1 Basic Purpose}
The rules and regulations in this part are designed to provide an amateur radio service having a fundamental purpose as expressed in the following principles:
\begin{enumerate}[A]
\scriptsize\item  Recognition and enhancement of the value of the amateur service to the public as a \emph{voluntary noncommercial communication service}, particularly with respect to providing emergency communications.\pause
\item Continuation and extension of the amateur's proven ability to contribute to the advancement of the radio art.\pause
\item Encouragement and improvement of the amateur service through rules which provide for advancing skills in both the communication and technical phases of the art.\pause
\item  Expansion of the existing reservoir within the amateur radio service of trained operators, technicians, and electronics experts. \pause
\item Continuation and extension of the amateur's unique ability to enhance international goodwill.
\end{enumerate}
\end{frame}

\begin{frame}
\frametitle{Keyphrase}
\begin{alltt}
\ldots\emph{a voluntary noncommercial\\communications service}\ldots
\end{alltt}
This phrase sums up almost every rule and tenant of amateur radio.
\end{frame}

\begin{frame}
\frametitle{A voluntary noncommercial communications service}
Noncommercial means no "pecuniary interest". It is illegal to profit from the use of amateur radio.\\
\hfil \\As with almost any rule there are exceptions"
\begin{itemize}
\item Teachers may use ham radio in the classroom as a teaching aid
\item "Code practice" transmissions
\item Disaster Drills
\end{itemize}
\end{frame}

\begin{frame}
\frametitle{More basic rules}
\begin{itemize}
\item No Music - expect transmission or re-transmission of a signal from a space station
\item No Broadcasting
\item No commercial traffic
\item No profanity
\item No codes or ciphers intended to hid content
\item No international third party traffic unless treaty-approved
\end{itemize}
\end{frame}

\subsection{licenses}

\begin{frame}
\frametitle{Licenses}
A license is valid for ten years, with a two year grace period. Upgrades don't count as renewals. Basic renewals are free!\\
There are five classes.
\begin{itemize}
\item *Novice
\item Technician
\item General
\item *Advanced
\item Extra
\end{itemize}
\end{frame}

\begin{frame}
\frametitle{Licenses}
There are four kinds of licenses, Individual hams hold both a "Station" and "Operator"
\begin{itemize}
\item Station
\item Operator
\item Club - W7OSU, K7CVO, W1AW
\item Special Event - A7W
\end{itemize}
Clubs can get a "club callsign", and events can get an event callsign.
\end{frame}

\subsection{Callsigns}

\begin{frame}
\frametitle{Callsigns}
\begin{itemize}
\item US callsigns start with A,K,N, or W
\item The format is one or two letters, a number, and one to three letters.
\item New callsigns are assigned in sequential order - number indicates the region in the US
\item Shorter callsigns are reserved for higher license classes
\item 1X1 for special events only
\end{itemize}
\end{frame}

\begin{frame}
\frametitle{Callsigns}
\begin{itemize}
\item KE7OSN
\item N8GFO \pause -Yep\pause
\item K7HZ \pause -That's an Extra\pause
\item VE6GLW \pause -That's Canadian, eh\pause
\item KLOO \pause -That's a commercial station\pause
\item WSJ509 \pause -Land Mobile, Benton County Sheriff\pause
\item Mission Base \pause -What is known as a "tactical callsign"\pause
\end{itemize}
\end{frame}

%\begin{frame}
%\frametitle{What IS an amateur station?}
%T1A10 (A) [97.3(a)(5)] What is the FCC Part 97 definition of an amateur station?\\ \pause \hfil \\
%A. A station in an Amateur Radio Service consisting of the apparatus necessary for carrying on radio
%communications
%\end{frame}

\begin{frame}
\frametitle{Operator}
Who "operates" an amateur station? \pause \\
The control operator, who is designated by the station licensee, and determines the privileges of operation.\\
e.g.\ if you are at a radio that can operate outside your privileges, you still can only use what you are licensed to.
\end{frame}

\begin{frame}
\frametitle{Your Callsign}
A station must transmit it's callsign at least every ten minutes and at the end of every communication.\\
Special situations have special rules\\
\begin{itemize}
\item Control operator working outside of a station licensee privileges.
\item Special event station control operator
\item Control operator using new privileges prior to FCC database update
\end{itemize}
\end{frame}

\begin{frame}
\frametitle{The Uniform Licensing System}
The ULS is an online database of FCC license information. A new licensee may use their privileges as soon as their information appears in the ULS. When you upgrade you may use your new privileges as soon as you pass the test.
\end{frame}

\begin{frame}
\frametitle{Typical uses of a callsign}
\begin{itemize}[<+->]
\item W7OSU This is KE7OSN
\item Net Control This is KE7OSN
\item This is W7OSU (Go Ahead)
\item CQ CQ CQ this is KE7OSN
\item KE7OSN monitoring
\item This is KF7FGE stroke (/) KE7OSN
\item Hey Bob, you around?
\end{itemize}
\end{frame}

\begin{frame}
\frametitle{Hey bob, you around}
Hey bob you around?\\
Legal? \\ \pause
Yes, as long as you keep to the every ten minutes and the end of every communication. \\ \pause
What if Bob isn't there? \\ \pause
KE7OSN clear
\end{frame}

\begin{frame}
\frametitle{Types of stations}
\begin{itemize}
\item Club – at least four  people, one of which accepts responsibility and is the “trustee”.
\item Space – at least 50km above the surface.
\item Beacon --  transmits a low-level signal for propagation studies
\item Repeater – retransmits  a signal heard on one frequency on another frequency.
\item Auxillary – a secondary receiver that feeds a repeater station.  
\end{itemize}
\end{frame}

\subsection{frequencies}

\begin{frame}
\frametitle{Band Plan}
\begin{center}
\includegraphics[width=\textwidth]{hambandscolor.pdf}
\end{center}
\end{frame}

\begin{frame}
\frametitle{ITU Band Names}
\begin{itemize}
\item MF - Medium Frequency 300KHz to 3MHz
\item HF - High Frequency 3MHz to 30 MHz
\item VHF - Very High Frequency 30MHz to 300MHz
\item UHF - Ultra High Frequency 300MHz to 3GHz
\item SHF - Super High Frequency 3GHz to 30GHz
\item EFE - Extremely High Frequency - 30GHz to 300GHz
\item THF - Tremendously High Frequency - 300GHZ to 3THz
\end{itemize}
\end{frame}

\begin{frame}
\frametitle{HF 3-30MHz}
	\begin{itemize}
	\item 80 Meters
		\begin{itemize}
		\item 3.525-3.600MHz: CW Only
		\end{itemize}
	\item 40 Meters
		\begin{itemize}
		\item 7.025-.125MHz: CW Only
		\end{itemize}
	\item 15 Meters
		\begin{itemize}
		\item 21.025-21.200MHz: CW Only
		\end{itemize}
	\item 10 Meters 
		\begin{itemize}
		\item  28.000-28.300MHz: CW, RTTY/Data 200 watts PEP max 
		\item 28.300-28.500MHz: CW, Phone 200 watts PEP max
		\end{itemize}
	\end{itemize}
\end{frame}

\begin{frame}
\frametitle{VHF 30-300MHz}
	\begin{itemize}
	\item 6 Meters
		\begin{itemize}
		\item 50.0-50.1MHz CW Only
		\item 50.1-54.0MHz All modes
		\end{itemize}
	\item 2 Meters
		\begin{itemize}
		\item  144.0-144.1MHz CW Only
		\item 144.1-148.0MHz All modes
		\end{itemize}
	\item 1.25 Meters
		\begin{itemize}
		\item 222.00-225.00MHz All modes
		\end{itemize}
	\end{itemize}
\end{frame}

\begin{frame}
\frametitle{UHF 300-3000MHz (3GHz)}
\begin{itemize}
\item 70 Centimeters
\begin{itemize}
\item 420.0-450.0MHz All Modes
\end{itemize}
\item 33 Centimeters
\begin{itemize}
\item 902.0-928.0MHz All Modes
\end{itemize}
\item 23 Centimeters
\begin{itemize}
\item 1240-1300MHz All Modes
\end{itemize}
\item 2.4GHz
\begin{itemize}
\item 2.3-2.31GHz
\item 2.39-2.45GHz *
\end{itemize}
\end{itemize}
\end{frame}

\begin{frame}
\frametitle{2.4GHz}
We share the 2390-2450MHz band with: 802.11 networks, cordless phones, video cameras, zigbee, etc.\ \\
We are PRIMARY users. We have first "rights". Secondary users must not cause us interference and must accept interference from our operations.
\end{frame}

\begin{frame}
\frametitle{SHF 3GHz-30GHz and up}
\begin{itemize}
\item 3.3-3.5GHz
\item  5.65-5.925GHz
\item  10.0-10.5GHz
\item  24.0-24.25GHz
\item  47.0-47.2GHz
\item  76.0-81.9GHz
\item  119.98-120.02GHz
\item  142-149GHz
\item  241-250GHz
\item  Everything above 300GHz
\end{itemize}
\end{frame}

\subsection{T1 Questions}

\begin{frame}
\frametitle{T1A}
\begin{itemize} %[<+->]
\tiny
\item\textbf{T1A01 For whom is the Amateur Radio Service intended?}\hfil \\D. Persons who are interested in radio technique solely with a personal aim and without pecuniary interest
\item\textbf{T1A02 What agency regulates and enforces the rules for the Amateur Radio Service in the United States?} \hfil \\ C. The FCC
\item\textbf{T1A03 Which part of the FCC rules contains the rules and regulations governing the Amateur Radio Service?}\hfil \\ D. Part 97
\item\textbf{T1A04 Which of the following meets the FCC definition of harmful interference?} \hfil \\ C. That which seriously degrades, obstructs, or repeatedly interrupts a radio communication service operating in accordance with the Radio Regulations
\item\textbf{T1A05 What is the FCC Part 97 definition of a space station?} \hfil \\  D. An amateur station located more than 50 km above the Earth’s surface
\item\textbf{T1A06 What is the FCC Part 97 definition of telecommand?}\hfil \\  C. A one-way transmission to initiate, modify or terminate functions of a device at a distance
\item\textbf{T1A07 What is the FCC Part 97 definition of telemetry?}\hfil \\ C. A one-way transmission of measurements at a distance from the measuring instrument
\item\textbf{T1A08 Which of the following entities recommends transmit/receive channels and other parameters for auxiliary and repeater stations?}\hfil \\B. Frequency Coordinator
\item\textbf{T1A09 Who selects a Frequency Coordinator?}\hfil \\ C. Amateur operators in a local or regional area whose stations are eligible to be auxiliary or repeater stations
\item\textbf{T1A10 What is the FCC Part 97 definition of an amateur station?}\hfil \\ A. A station in an Amateur Radio Service consisting of the apparatus necessary for carrying on radio communications
\item\textbf{T1A11 Which of the following stations transmits signals over the air from a remotereceive site to a repeater for retransmission?} \hfil \\C. Auxiliary station
\end{itemize}
\end{frame}

\begin{frame}
\frametitle{T1B}
\begin{itemize} %[<+->]
\tiny
\item\textbf{T1B01 What is the ITU?} \hfil \\ B. A United Nations agency for information and communication technology issues
\item\textbf{T1B02 North American amateur stations are located in which ITU region?} \hfil \\ B. Region 2
\item\textbf{T1B03 Which frequency is within the 6 meter band?} \hfil \\ B. 52.525 MHz
\item\textbf{T1B04 Which amateur band are you using when your station is transmitting on 146.52MHz?} \hfil \\ A. 2 meter band
\item\textbf{T1B05 Which 70 cm frequency is authorized to a Technician Class license holder operating in ITU Region 2? }\hfil \\ C. 443.350 MHz
\item\textbf{T1B06 Which 23 cm frequency is authorized to a Technician Class operator license?} \hfil \\ B. 1296 MHz
\item\textbf{T1B07 What amateur band are you using if you are transmitting on 223.50 MHz?} \hfil \\ D. 1.25 meter band
\item\textbf{T1B08 What do the FCC rules mean when an amateur frequency band is said to be available on a secondary basis?} \hfil \\ C. Amateurs may not cause harmful interference to primary users
\item\textbf{T1B09 Why should you not set your transmit frequency to be exactly at the edge of an amateur band or sub-band?} \hfil \\ A. To allow for calibration error in the transmitter frequency display \hfil \\ B. So that modulation sidebands do not extend beyond the band edge \hfil \\ C. To allow for transmitter frequency drift \hfil \\ D. All of these choices are correct
\item\textbf{T1B10 Which of the bands available to Technician Class operators have mode-restricted sub-bands?} \hfil\\ C. The 6 meter, 2 meter, and 1.25 meter bands
\item\textbf{T1B11 What emission modes are permitted in the mode-restricted sub-bands at 50.0 to 50.1 MHz and 144.0 to 144.1 MHz? }\hfil \\ A. CW only
\end{itemize}
\end{frame}

\begin{frame}
\frametitle{T1C}
\begin{itemize}
\tiny
\item\textbf{T1C01 Which type of call sign has a single letter in both the prefix and suffix?}\hfil\\ C. Special event
\item\textbf{T1C02 Which of the following is a valid US amateur radio station call sign?}\hfil\\ B. W3ABC
\item\textbf{T1C03 What types of international communications are permitted by an FCC-licensed amateur station?} \hfil\\A. Communications incidental to the purposes of the amateur service and remarks of a personal character
\item\textbf{T1C04 When are you allowed to operate your amateur station in a foreign country?}\hfil\\ A. When the foreign country authorizes it
\item\textbf{T1C05 What must you do if you are operating on the 23 cm band and learn that you are interfering with a radiolocation  station outside the United States?} \hfil\\A. Stop operating or take steps to eliminate the harmful interference
\item\textbf{T1C06 From which of the following may an FCC-licensed amateur station transmit, in addition to places where the FCC regulates communications?} \hfil\\D. From any vessel or craft located in international waters and documented or registered in the United
States
\item\textbf{T1C07 What may result when correspondence from the FCC is returned as undeliverable because the grantee failed to provide the correct mailing address?} \hfil\\B. Revocation of the station license or suspension of the operator license
\item\textbf{T1C08 What is the normal term for an FCC-issued primary station/operator license grant?}\hfil\\ C. Ten years
\item\textbf{T1C09 What is the grace period following the expiration of an amateur license within which the license may be renewed?} \hfil\\A. Two years
\item\textbf{T1C10 How soon may you operate a transmitter on an amateur service frequency after you pass the examination required for your first amateur radio license?} \hfil\\C. As soon as your name and call sign appear in the FCC s ULS database
\item\textbf{T1C11 If your license has expired and is still within the allowable grace period, may you continue to operate a transmitter on amateur service frequencies?} \hfil\\A. No, transmitting is not allowed until the ULS database shows that the license has been renewed
\end{itemize}
\end{frame}

\begin{frame}
\frametitle{T1D}
\begin{itemize}
\tiny
\item\textbf{T1D01 With which countries are FCC-licensed amateur stations prohibited from exchanging communications?} \hfil\\A. Any country whose administration has notified the ITU that it objects to such communications
\item\textbf{T1D02 On which of the following occasions may an FCC-licensed amateur station exchange messages with a U.S. military station?} \hfil\\A. During an Armed Forces Day Communications Test
\item\textbf{T1D03 When is the transmission of codes or ciphers allowed to hide the meaning of a message transmitted by an amateur station?} \hfil\\C. Only when transmitting control commands to space stations or radio control craft
\item\textbf{T1D04 What is the only time an amateur station is authorized to transmit music?} \hfil\\A. When incidental to an authorized retransmission of manned spacecraft communications
\item\textbf{T1D05 When may amateur radio operators use their stations to notify other amateurs of the availability of equipment for sale or trade?} \hfil\\A. When the equipment is normally used in an amateur station and such activity is not conducted on a regular basis
\item\textbf{T1D06 Which of the following types of transmissions are prohibited? }\hfil\\A. Transmissions that contain obscene or indecent words or language
\item\textbf{T1D08 When may the control operator of an amateur station receive compensation for operating the station?} \hfil\\B. When the communication is incidental to classroom instruction at an educational institution
\item\textbf{T1D09 Under which of the following circumstances are amateur stations authorized to transmit signals related to broadcasting, program production, or news gathering, assuming no other means is available?}\hfil\\ A. Only where such communications directly relate to the immediate safety of human life or protection of property
\item\textbf{T1D10 What is the meaning of the term broadcasting in the FCC rules for the amateur services?} \hfil\\D. Transmissions intended for reception by the general public
\item\textbf{T1D11 Which of the following types of communications are permitted in the Amateur Radio Service?} \hfil\\A. Brief transmissions to make station adjustments
\end{itemize}
\end{frame}

\begin{frame}
\frametitle{T1E}
\begin{itemize}
\tiny
\item\textbf{T1E01 When must an amateur station have a control operator?}\hfil\\A. Only when the station is transmitting
\item\textbf{T1E02 Who is eligible to be the control operator of an amateur station?} \hfil\\D. Only a person for whom an amateur operator/primary station license grant appears in the FCC database or who is authorized for alien reciprocal operation
\item\textbf{T1E03 Who must designate the station control operator?} \hfil\\A. The station licensee
\item\textbf{T1E04 What determines the transmitting privileges of an amateur station?}\hfil\\D. The class of operator license held by the control operator
\item\textbf{T1E05 What is an amateur station control point?} \hfil\\C. The location at which the control operator function is performed
\item\textbf{T1E06 Under which of the following types of control is it permissible for the control operator to be at a location other than the control point?} \hfil\\B. Automatic control
\item\textbf{T1E07 When the control operator is not the station licensee, who is responsible for the proper operation of the station?} \hfil\\D. The control operator and the station licensee are equally responsible
\item\textbf{T1E08 What type of control is being used for a repeater when the control operator is not present at a control point?}\hfil\\C. Automatic control
\item\textbf{T1E09 What type of control is being used when transmitting using a handheld radio?} \hfil\\D. Local control
\item\textbf{T1E10 What type of control is used when the control operator is not at the station location but can indirectly manipulate the operating adjustments of a station?} \hfil\\B. Remote
\item\textbf{T1E11 Who does the FCC presume to be the control operator of an amateur station, unless documentation to the contrary is in the station records?}\hfil\\D. The station licensee
\end{itemize}
\end{frame}

\begin{frame}
\frametitle{T1F}
\begin{itemize}
\tiny
\item\textbf{T1F01 What type of identification is being used when identifying a station on the air as Race Headquarters ?}\hfil\\ A. Tactical call
\item\textbf{T1F02 When using tactical identifiers, how often must your station transmit the stations FCC-assigned call sign?} \hfil\\C. Every ten minutes
\item\textbf{T1F03 When is an amateur station required to transmit its assigned call sign?}\hfil\\D. At least every 10 minutes during and at the end of a contact
\item\textbf{T1F04 Which of the following is an acceptable language for use for station identification when operating in a phone sub-band?} \hfil\\C. The English language
\item\textbf{T1F05 What method of call sign identification is required for a station transmitting phone signals?} \hfil\\B. Send the call sign using CW or phone emission
\item\textbf{T1F06 Which of the following formats of a self-assigned indicator is acceptable when identifying using a phone transmission?}\hfil\\D. All of these choices are correct
\item\textbf{T1F07  Which of the following restrictions apply when appending a self-assigned call sign indicator?} \hfil\\D. It must not conflict with any other indicator specified by the FCC rules or with any call sign prefix assigned to another country
\item\textbf{T1F08 When may a Technician Class licensee be the control operator of a station operating in an exclusive Extra Class operator segment of the amateur bands?} \hfil\\A. Never
\item\textbf{T1F09 What type of amateur station simultaneously retransmits the signal of another amateur station on a different channel or channels?} \hfil\\C. Repeater station
\item\textbf{T1F10 Who is accountable should a repeater inadvertently retransmit communications that violate the FCC rules?} \hfil\\A. The control operator of the originating station
\item\textbf{T1F11 To which foreign stations do the FCC rules authorize the transmission of non-emergency third party communications?} \hfil\\A. Any station whose government permits such communications
\item\textbf{T1F12 How many persons are required to be members of a club for a club station license to be issued by the FCC?} \hfil\\B. At least 4
\item\textbf{T1F13 When must the station licensee make the station and its records available for FCC inspection?}\hfil\\ B. Any time upon request by an FCC representative
\end{itemize}
\end{frame}
\section{Operating Practices}

\subsection{Emergency Operations}
\begin{frame}
\frametitle{97.403 Safety of life and protection of property}
No provision of these rules prevents the use by an amateur station of any means of radio communication at its disposal to provide essential communication needs in connection with the immediate safety of human life and immediate protection of property \textbf<2->{when normal communication systems are not available.}
\end{frame}

\begin{frame}
\frametitle{ARES}
ARES – \textbf{A}mateur \textbf{R}adio \textbf{E}mergency \textbf{S}ervice
\begin{itemize}
\item Organized and run by ARRL
\item Supports governmental and NGO groups.
\item Most groups are organized at the county level
\item “EC” – Emergency Coordinator
\end{itemize}
\end{frame}

\begin{frame}
\frametitle{RACES}
RACES – \textbf{R}adio \textbf{A}mateur \textbf{C}ivil \textbf{E}mergency \textbf{S}ervice
\begin{itemize}
\item Defined by the FCC
\item Supports governmental agencies ONLY.
\item Operators are registered with the controlling agency. 
\item RACES Officer
\item Activated by federal declaration of emergency. 
\item In Oregon, ARES members are also registered in RACES
\end{itemize}
\end{frame}

\subsection{Nets and messages}
\begin{frame}
\frametitle{Disaster == Organized Chaos}
To keep some organization to the use of frequencies and communications,  groups are organized into “nets”.

A “net” is a group of stations that are cooperating in the use of a frequency. The “net control” is responsible for deciding who gets to talk. 
\end{frame}

\begin{frame}
\frametitle{Disaster == Organized Chaos}
There are two kinds of nets:
 \begin{description}
\item[Directed] - the net control is strict in controlling who talks to whom.  Stations tell net control they have a message for another station, and the net control directs them to call that station and pass the message.
\item[Free] - the net control allows stations to contact each other as they need to.   
\end{description}
\end{frame}

\begin{frame}
\frametitle{Disaster == Organized Chaos}
There are three types of messages
\begin{description}
\item[Formal] - written messages.
\item[Informal] - unwritten messages.
\item[Administrative] - station to station housekeeping.
\end{description}
\end{frame}

\begin{frame}
\frametitle{Written Messages – Formal Traffic}
ARES and RACES have adopted the NIMS/ICS system for written traffic. I.e., ICS-213 message forms, in either digital or transcribed versions.\\ \hfil \\
 
\uncover<2->{The TEST doesn’t ask you about that. The TEST deals with the National Traffic System message form.}
\end{frame}

\begin{frame}
\frametitle{Written Messages – Formal Traffic}
The National Traffic System (NTS) is a system organized by ARRL to transmit messages in a standard format, usually concerning “Health and Welfare”. For example: “Aunt Martha arrived home safely. Have a happy birthday.”  Or “welcome to Ham radio”. \\

These messages use the NTS RadioGram form. The process is described in depth in the Message Processing Guidelines (MPG).
\end{frame}

\begin{frame}
\frametitle{NTS RadioGram Form}
\centering \includegraphics[width=\textwidth]{radiogramform.png}
\end{frame}

\subsection{Radio Speak}
\begin{frame}
\frametitle{Types of radio short-hand}
Amateur radio has its own codes, and slang. Much like 1337 or txt, this "shared language" makes it easier to communicate quickly, and efficently. Much of it comes from the days of telegraph and Morse Code.
\begin{description}
\item[Q Codes] - Three letter codes beginning with Q
\item[Number codes] - Codes sent as numbers, we really only us 73
\item[Pro-words] - Standardized ways of saying things in a clear and concise fashion
\item[Phonetics] - Words for letters, try saying BCDEZGT five times fast.
\end{description}
\end{frame}

\begin{frame}
\frametitle{Q-Codes}
Q codes are three letter codes that begin with Q and not QU and can be sent as either a question or a response. Really useful when using Morse Code as the codes are much shorter than what they represent. Some common Q codes are listed below.
\begin{description}
\item[QSY] Change frequency
\item[QRT] Stop transmitting
\item[QRZ] I'm calling
\item[QR\textbf<2->{M}] \textbf<2->{M}an made interference
\item[QR\textbf<2->{N}] \textbf<2->{N}atural interference or \textbf<2->{N}oise
\item[QSL] Acknowledge
\item[QST] Message to all amateurs
\end{description}
\end{frame}

\begin{frame}
\frametitle{Pro-words}
Pro or Professional words are used as shorthand, and because they prevent confusion. Yea and Nah kinda sound the same.
\begin{description}
\item[Roger] Received
\item[WilCO] Will Comply
\item[Over] I'm done talking for now
\item[Out] I'm done talking to you
\item[This Is] I'm going to say my callsign now
\item[Wait] Hold on for a while
\item[Affirmative] Yes
\item[Negative] No
\end{description}
\end{frame}

\begin{frame}
\frametitle{Phonetics}
In amateur radio we use ITU phonetics, this helps us reduce the potential of confusion over letters that sound the same
\begin{tabular}{l l l}
\scriptsize A - Alfa "AL-FAH"& \scriptsize  N - November "NO-VEM-BER" & \scriptsize  0 "ZEE-RO"\\
\scriptsize B - Bravo "BRAH-VOH"& \scriptsize  O - Oscar "OSS-CAH" & \scriptsize  1 "WUN"\\
\scriptsize C - Charlie "CHAR-LEE"& \scriptsize  P - Papa "PAH-PAH" & \scriptsize  2 "TOO"\\
\scriptsize D - Delta "DELL-TAH"& \scriptsize  Q - Quebec "KEH-BECK" & \scriptsize  3 "TH-UH-REE"\\
\scriptsize E - Echo "ECK-OH"& \scriptsize  R - Romeo "ROW-ME-OH & \scriptsize  4 "FOW-ER"\\
\scriptsize F - Foxtrot "FOKS-TROT"& \scriptsize  S - Sierra "SEE-AIR-RAH" & \scriptsize  5 "FI-IV" OR "FIFE"\\
\scriptsize G - Golf "GOLF"& \scriptsize  T - Tango "TANG-GO" & \scriptsize  6 "SIX"\\
\scriptsize H - Hotel "HOH-TELL"& \scriptsize  U - Uniform "YOU-NEE-FORM" & \scriptsize  7 "SEV-EN"\\
\scriptsize I - India "IN-DEE-AH"& \scriptsize  V - Victor "VIK-TAH" & \scriptsize  8 "ATE"\\
\scriptsize J - Juliett "JEW-LEE-ETT& \scriptsize  W - Whiskey "WISS-KEY" & \scriptsize  9 "NIN-ER"\\
\scriptsize K - Kilo "KEE-LOH"& \scriptsize  X - X-Ray "ECKS-RAY"\\
\scriptsize L - Lima "LEE-MAH& \scriptsize  Y - Yankee "YANG-KEY"\\
\scriptsize M - Mike "MIKE"& \scriptsize  Z - Zulu "ZOO-LOO"\\
\end{tabular}
W7QH becomes "Whiskey 7 Quebec Hotel" 
\end{frame}

\begin{frame}
\frametitle{CQ and 73}
There are two other special cases.
\begin{itemize}
\item CQ is the standard calling call. Think of it as Seek You, though no one really knows where it comes from. It is common to add extra stuff depending on the situation. You might hear CQ JOTA, CQ Field Day, CQ Contest, CQ DX, CQ Oregon. This lets people pick who they are looking for. A common general CQ would sound like "CQ CQ CQ this is KE7OSN calling CQ CQ CQ"
\item The other thing that comes up is the number 73, this goes back to the old Western Union Telegraph 92 codes, these were numbers that could be used in place of certian phrases, most of them dealing with packages or trains. 73 means "Beast Regards" and is generally used as "goodbye"
\end{itemize}
\end{frame}

\subsection{T2 Questions}

\begin{frame}
\frametitle{T2A}
hello world
\end{frame}

\begin{frame}
\frametitle{T2B}
hello world
\end{frame}

\begin{frame}
\frametitle{T2C}
hello world
\end{frame}

\begin{frame}
\frametitle{Title}
hello world
\end{frame}

\begin{frame}
\frametitle{Title}
hello world
\end{frame}

\begin{frame}
\frametitle{Title}
hello world
\end{frame}

\begin{frame}
\frametitle{Title}
hello world
\end{frame}

\begin{frame}
\frametitle{Title}
hello world
\end{frame}

\begin{frame}
\frametitle{Title}
hello world
\end{frame}

\begin{frame}
\frametitle{Title}
hello world
\end{frame}

\begin{frame}
\frametitle{Title}
hello world
\end{frame}

\begin{frame}
\frametitle{Title}
hello world
\end{frame}

\begin{frame}
\frametitle{Title}
hello world
\end{frame}

%
%\begin{frame}
%\frametitle{Title}
%hello world
%\end{frame}
%
\end{document}